
\documentclass{article}
\usepackage{amssymb, latexsym, amsmath, amsthm, amsfonts, amsbsy, enumerate, tabularx, graphicx}
\newtheorem{definicion}{Definici\'on}

\newtheorem{teorema}{Teorema}
\newtheorem{prop}{Proposici\'on}
\newtheorem{coro}{Corolario}
\newtheorem{lema}{Lema}
\theoremstyle{definition}
\newtheorem{nota}{Nota}
\newtheorem{notas}{Notas}
\newtheorem{ejemplo}{Ejemplo}
\newtheorem{ejemplos}{Ejemplos}
%\newtheorem{definicion}{Definici\'on}
\newtheorem{notacion}{Notaci\'on}
\newcommand{\sen}{\operatorname{sen}}
\newcommand{\senh}{\operatorname{senh}}
\newcommand{\csch}{\operatorname{csch}}
\newcommand{\sech}{\operatorname{sech}}
\newcommand{\arcsec}{\operatorname{arcsec}}
\newcommand{\arcsen}{\operatorname{arcsen}}
\newcommand{\id}{\operatorname{id}}
\begin{document}




  \section*{T14 }

	\begin{enumerate}
	\item Para las funciones dadas, encuentra los puntos cr\'iticos y determina si son 
			 m\'aximos locales, m\'inimos locales o puntos silla.
			 
			 \begin{enumerate}
			 \item $f(x,y)=y+x\sen(y)$,
			 \item $f(x,y)=x^2+(y-1)^2$,
			 \item $f(x,y)=x^3+y^3-3xy$.
			\end{enumerate}			 	


	\item Hallar el punto en el plano $2x-y+2z=20$ m\'as cercano al origen.
	
	\item Hallar los valores m\'aximos y m\'inimos absolutos de la funci\'on $f(x,y)=ax^2+by^2-1$,
	definida en $ax^2+by^2 \leq 1$, (donde $a$ y $b$ son constantes positivas).
	
      \item Considera la funci\'on $f(x,y)=3x^4-4x^2y+y^2$.
        \begin{enumerate}
        \item Prueba que, para cada l\'inea por el origen $y=mx$, la funci\'on
          alcanza un m\'inimo sobre dicha l\'inea.
        \item Prueba que no existe un m\'inimo local en $(0,0)$.
        \item Haz un bosquejo de los puntos $(x,y)$ en el plano para los cuales
          $f(x,y)>0$ y para los cuales $f(x,y)<0$.
        \end{enumerate}


      \item Determina los m\'aximos, m\'inimos relativos y absolutos, as\'i como los
        puntos silla de la funci\'on $f(x,y)=xy(1-x^2-y^2)$ en el cuadrado $0\leq x \leq 1$,
        $0 \leq y \leq 1$.

      \item (M\'etodo de m\'inimos cuadrados)

        Dados $n$ n\'umeros distintos, $x_1,\dots, x_n$ y otros $n$ n\'umeros $y_1,\dots y_n$
        (no necesariamente distintos), por lo general no es posible encontrar una funci\'on de la
        forma $f(x)=ax+b$, que satisfaga $f(x_i)=y_i$, para toda $i$. Sin embargo se puede
        tratar de encontrar una funci\'on que minimize el error cuadrado
        $$
        E(a,b)=\sum_{i=1}^n (f(x_i)-y_i)^2
        $$

        Determinar los valores de $a$ y $b$ que hagan esto.

      \item Encuentra $c>0$ tal que la funci\'on $f(x,y)=x^2+xy+cy^2$ tiene un
        punto silla en $(0,0)$.

      \item Halla los valores m\'aximos y m\'inimos absolutos
        para $f(x,y)=\sen(x)+\cos(y)$, en el rect\'angulo
        $[0,2\pi]\times [0, 2\pi]$.

      \item Sean $z_1,\dots, z_k $, $k$ puntos distintos en $\mathbb{R}^n$.
        Para $x\in \mathbb{R}^n$ define
        $$
        f(x)=\sum_{j=1}^k \| x- z_k\|^2
        $$

        Prueba que $f$ alcanza su m\'inimo en el punto
        $\frac{1}{k}\sum_{j=1}^k z_j$ (conocido como el centroide de los
        puntos $z_1,\dots, z_k$).
        
	\end{enumerate}	
	
  
       \end{document}
