
\documentclass{article}
\usepackage{amssymb, latexsym, amsmath, amsthm, amsfonts, amsbsy, enumerate, tabularx, graphicx}
\newtheorem{definicion}{Definici\'on}

\newtheorem{teorema}{Teorema}
\newtheorem{prop}{Proposici\'on}
\newtheorem{coro}{Corolario}
\newtheorem{lema}{Lema}
\theoremstyle{definition}
\newtheorem{nota}{Nota}
\newtheorem{notas}{Notas}
\newtheorem{ejemplo}{Ejemplo}
\newtheorem{ejemplos}{Ejemplos}
%\newtheorem{definicion}{Definici\'on}
\newtheorem{notacion}{Notaci\'on}
\newcommand{\sen}{\operatorname{sen}}
\newcommand{\senh}{\operatorname{senh}}
\newcommand{\csch}{\operatorname{csch}}
\newcommand{\sech}{\operatorname{sech}}
\newcommand{\arcsec}{\operatorname{arcsec}}
\newcommand{\arcsen}{\operatorname{arcsen}}
\newcommand{\id}{\operatorname{id}}
\begin{document}




  \section*{T12 }
  \begin{enumerate}
  

    
  \item Considera las funciones $F:\mathbb{R}^3\to \mathbb{R}^2$ y $G:\mathbb{R}^3\to \mathbb{R}^3$ dadas
    por
    $$
    F(x,y,z)=(x^2+y+z, 2x+y+z^2), \quad G(u,v,w)=(2uv^2w^2,w^2\sen(v),u^2e^v)
    $$
    \begin{enumerate}
    \item Encuentra la matriz de derivadas parciales $D_{(x,y,z)}F$ y $D_{(u,v,w)}G$.
    \item Define $H=F\circ G$. Usa la regla de la cadena para calcular la matriz derivada
      parciales $D_{(u,0,w)}H$.
    \end{enumerate}


    
  \item Hallar la ecuaci\'on del plano tangente a las superficies
    dadas en los puntos indicados.
    \begin{enumerate}
    \item $x^2+2y^2+3xz=10$ en $(1,2,1/3)$,
    \item $y^2-x^2=3$ en $(1,2,8)$,
    \item $xyz=1$ en $(1,1,1)$.
    \end{enumerate}


  %\item Dada una superficie $S$ en $\mathbb{R}^3$ y un punto $p_0$ en la superficie, vamos a
   % denotar $N,U,L$, a tres vectores unitarios con las caracter\'isticas de que: $N$
   % es el vector normal al plano tangente a $S$ que pasa por $p_0$; $U$ es vector que indica
   % la direcci\'on donde la raz\'on de crecimiento, en $p_0$, es m\'axima; $L$ es el
   % vector donde la raz\'on de crecimiento, en $p_0$, es cero.

    %Nota:  en general hay dos elecciones para dichos vectores, pues podemos tomar su negativo
    %(por ejemplo $-N$ o $N$).

    %Para las siguientes superficies calcula $N,U$ y $L$ para un punto general de la superficie.
    %\begin{enumerate}
    %\item $S$ es la gr\'afica de la funci\'on $g(x,y)=1-x-y$ (un plano).
    %\item $S$ es la superficie dada por la ecuaci\'on $x^2+y^2-z^2=0$ (un cono). En este ejemplo
    %  se evita el v\'ertica, pues el plano tangente en ese punto no est\'a bien definido.
    %\end{enumerate}


  \item Recuerda que, para una superficie de nivel $S \subset \mathbb{R}^3$, de la funci\'on $g(x,y,z)$,
    la ecuaci\'on del plano tangente es
    $$
    \langle \nabla_{(x_0,y_0,z_0)}g, (x-x_0,y-y_0,z-z_0) \rangle =0
    $$
    donde $(x_0,y_0z_0)$ es un punto en la superficie $S$.

    Demuestra que, como caso especial, la f\'ormula del plano tangente a la gr\'afica de la funci\'on
    $f(x,y)$, se puede obtener de la ecuaci\'on anterior si se considera a la gr\'afica
    como una superficie de nivel de $F(x,y,z)=f(x,y)-z$.
 
    
  \item Considera la funci\'on $f(x,y)=-(1-x^2-y^2)^{1/2}$, definida para los puntos $(x,y)$
    con $x^2+y^2<1$. Prueba que el plano tangente a la gr\'afica de $f$, en el punto $(x_0,y_0,f(x_0,y_0))$
    es ortogonal al vector $(x_0,y_0,f(x_0,y_0))$.

    \item Usa la regla de la cadena para probar que, si $g:U\to \mathbb{R}$ es de clase
      $C^1$ en $U$, con $U$ un abierto de $\mathbb{R}^n$, entonces
      $$
      \nabla_{p}(1/g)=-\frac{1}{g(p)^2}\nabla_p g
      $$    
      donde suponemos que $g(p)\ne 0$, para toda $p\in U$.
    
  \item Sean $G:\mathbb{R}^m \to \mathbb{R}^n$, una funci\'on de clase $C^1$ en $\mathbb{R}^m$, con funciones
    coordenadas $G(q)=(g_1(q),\dots, g_n(q))$ y sea $f:\mathbb{R}^n \to \mathbb{R}$ una funci\'on
    de clase $C^1$ en $\mathbb{R}^n$ y sea $h=f\circ G$. Usa la regla de la cadena para demostrar que
    el gradiente de $h$ es una combinaci\'on lineal de los gradientes de las $g_k$, en espec\'ifico:
    $$
    \nabla_{q_0} h= \sum_{k=1}^n \partial_{p_k}f(G(q_0)) \nabla_{q_0}g_k
    $$
    nota que $\partial_{p_k}f(g(q_0))$ es escalar y $\nabla_{q_0}g_k$ es vector.


    \item Encuentra el conjunto de puntos $(a,b,c)$ en $\mathbb{R}^3$, para los cuales las
      dos esferas: $(x-a)^2+(y-b)^2+(z-c)^2=1$ y $x^2+y^2+z^2=1$, se intersectan ortogonalmente.

      Nota: Dos superficies se intersectan ortogonalmente si, para todo punto en su intersecci\'on,
      los planos tangentes son ortogonales. 

      Sugerencia: ve las esferas como superficies de nivel y utiliza gradientes. 

  \item
    \begin{enumerate}
    \item Considera la funci\'on $I:\mathbb{R}^3\to \mathbb{R}^3$ dada por $I(x,y,z)=(x,y,z)$.
      Demuestra que
      $$
      D_{(x,y,z)}I=\left[
        \begin{array}{ccc}
          1 & 0 & 0 \\
          0 & 1 & 0 \\
          0 & 0 & 1
        \end{array}
      \right]
      $$
    \item Encuentra todas las funciones diferenciables en $\mathbb{R}^3$,
      $F:\mathbb{R}^3 \to \mathbb{R}^3$, para las cuales

      $$
      D_{(x,y,z)}F=\left[
        \begin{array}{ccc}
          x & 0 & 0 \\
          0 & y & 0 \\
          0 & 0 & x
        \end{array}
      \right]
      $$

    \item Sean $p,q,r:\mathbb{R}\to \mathbb{R}$ funciones continuas en todo $\mathbb{R}$. Encuentra
      todas la funciones diferenciables en $\mathbb{R}^3$, $G:\mathbb{R}^3 \to \mathbb{R}^3$, para
      las cuales

      $$
      D_{(x,y,z)}G=\left[
        \begin{array}{ccc}
          p(x) & 0 & 0 \\
          0 & q(y) & 0 \\
          0 & 0 & r(z)
        \end{array}
      \right]
      $$

      
    \end{enumerate}

    \end{enumerate}

       \end{document}


