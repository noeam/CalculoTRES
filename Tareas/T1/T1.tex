
\documentclass{article}
\usepackage{amssymb, latexsym, amsmath, amsthm, amsfonts, amsbsy, enumerate, tabularx, graphicx}
\newtheorem{definicion}{Definici\'on}

\newtheorem{teorema}{Teorema}
\newtheorem{prop}{Proposici\'on}
\newtheorem{coro}{Corolario}
\newtheorem{lema}{Lema}
\theoremstyle{definition}
\newtheorem{nota}{Nota}
\newtheorem{notas}{Notas}
\newtheorem{ejemplo}{Ejemplo}
\newtheorem{ejemplos}{Ejemplos}
%\newtheorem{definicion}{Definici\'on}
\newtheorem{notacion}{Notaci\'on}
\newcommand{\sen}{\operatorname{sen}}
\newcommand{\senh}{\operatorname{senh}}
\newcommand{\csch}{\operatorname{csch}}
\newcommand{\sech}{\operatorname{sech}}
\newcommand{\arcsec}{\operatorname{arcsec}}
\newcommand{\arcsen}{\operatorname{arcsen}}
\begin{document}




  \section*{T1}

	\begin{enumerate}

		\item Da un ejemplo que muestra que $\|p+q\|\ne \|p\|+\|q\|$.	


 \item Este ejercicio muestra que el \'area del tri\'angulo con v\'ertices  $0, u=(u_1,u_2)$ y $v=(v_1,v_2)$
       es $\frac{1}{2}|u_1v_2-u_2v_1|$.
       \begin{enumerate}
       \item Prueba que el \'area del tri\'angulo es $\frac{1}{2}\|u\|(\|v\| \sen(\theta))$,
         donde $\theta\in [0,\pi]$, es el \'angulo entre $v$ y $u$.
       \item Prueba que $\sen(\theta)=\sqrt{1- \left( \frac{\langle u , v \rangle }{\|u\|\|v\|} \right)^2}$.
       Sugerencia: usa ecuaci\'on $\langle u, v \rangle=\|u\|\|v\|\cos(\theta)$.
       \item Prueba que el \'area es igual a $\frac{1}{2}\sqrt{\|u\|^2\|v\|^2-(\langle u,  v\rangle )^2}$.
       \item Finalmente prueba que $\|u\|^2\|v\|^2-(\langle u,  v\rangle )^2=(u_1v_2-u_2v_1)^2$.
       \end{enumerate}
       
        \item Encuentra el \'area del tri\'angulo con v\'ertices en los puntos $(1,1), (0,3)$ y $(4,5)$.
	
	
         \begin{definicion}
           Dos vectores $p,q\in \mathbb{R}^n$ se llaman ortogonales si $\langle p,q \rangle=0$.
         \end{definicion}
         
         
       \item (Teorema de Pit\'agoras)

         Par dos vectores $p, q \in \mathbb{R}^n$, demuestra:

         $\|p+q\|^2=\|p\|^2+\|q\|^2$ si y s\'olo si los vectores son prependiculares.

 \item (La otra desigualdad del tri\'angulo)

                Demuestra, para todos  los vectores $p,q$ en un espacio con producto interior:
                $$
                |\|p\| - \|q\|| \leq \|p-q\|.
                $$

                Sugerencia: empieza con $\|p\|$ suma  y resta $q$ y usa la desigualdad del tri\'angulo.	

\item\label{Ejer:DeterRenglonesRepetidos} Sea $A$ una matriz de $3 \times 3$. Supongamos que dos renglones de $A$ son iguales. Demuestra
que $\det(A)=0$. Sugerencia: utiliza la propiedad alternante del determinante.

\item Considera los vectores $v=(v_1,v_2,v_3), w=(w_1,w_2,w_3)$. 

Dados escalares $\alpha, \beta$ construye
el vector $u=\alpha v+ \beta w$. Demuestra que 
$$
\det\left[
\begin{array}{ccc}
u_1 & u_2 & u_3 \\
v_1 & v_2 & v_3 \\
w_1 & w_2 & w_3
\end{array}
\right]=0
$$


\item Demuestra 
	$$
	(u\times v) \times w= (u \cdot w) v - (v \cdot w ) u
	$$
	
	$$
	u\times (v\times w)=   (u \cdot w )v -(u \cdot v ) w	
	$$

	Usa las f\'ormulas anteriores para dar un ejemplo de vectores $u,v$ y $w$ que muestren 
	$(u\times v) \times w\ne u \times (v \times w) $.
	
	\end{enumerate}





  
       \end{document}
