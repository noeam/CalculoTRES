\documentclass[12pt]{article}
\setlength{\topmargin}{-.75in}\addtolength{\textheight}{2.00in}
\setlength{\oddsidemargin}{.00in}\addtolength{\textwidth}{.75in}
\usepackage{amsmath,amssymb}
\nofiles

\pagestyle{empty}
\setlength{\parindent}{0in}
\newcommand{\sen}{\operatorname{sen}}


\begin{document}

\noindent{\sc {\bf{\large Quiz 5 }}
           \hfill C\'alculo 3, semestre 2020-2}
\bigskip

\noindent {\sc{
            { \Large Nombre: \underline {\hspace {10 cm }}}}}
            
\bigskip
\bigskip
\bigskip


\begin{enumerate}

\item (2 pts) Calcula las siguientes derivadas parciales
  en los puntos indicados.

  \begin{enumerate}
  \item(0.5 pts) $f(x,y)=e^{2x^2+y^3}$. Calcular $\partial_xf(1,2)$.
  \item (0.5 pts) $f(r,\theta)=re^{\sen(\theta)}$.
      Calcular $\partial_\theta f(1,\pi)$.
    \item (0.5 pts) $f(x,y,z)=3x^{2}y^{3}z^{5}$. Calcular $\partial_zf(2,1,6)$.
    \item (0.5 pts) $f(a,b,c)=\sen(a)\sen(a+b)\sen(a+b+c)$.
      Calcular $\partial_cf(\pi/4,\pi/4, \pi/4)$.
    \end{enumerate}
    
    
  
\vspace{4cm}  
  
\item (3 pts) Considera la funci\'on

  \begin{equation*}
    f(x,y)=\left\{
      \begin{array}{cc}
        \frac{2xy^2}{x^2+y^4} & (x,y)\ne(0,0)\\
        0 & (x,y)=(0,0)
      \end{array}
    \right.
    \end{equation*}
    Usando la definici\'on de derivadas parciales usando l\'imites,
    prueba que $\partial_xf(0,0)$ y $\partial_yf(0,0)$ existen
    y calcula sus valores.



  \end{enumerate}


  
\end{document}
