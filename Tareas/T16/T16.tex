
\documentclass{article}
\usepackage{amssymb, latexsym, amsmath, amsthm, amsfonts, amsbsy, enumerate, tabularx, graphicx}
\newtheorem{definicion}{Definici\'on}

\newtheorem{teorema}{Teorema}
\newtheorem{prop}{Proposici\'on}
\newtheorem{coro}{Corolario}
\newtheorem{lema}{Lema}
\theoremstyle{definition}
\newtheorem{nota}{Nota}
\newtheorem{notas}{Notas}
\newtheorem{ejemplo}{Ejemplo}
\newtheorem{ejemplos}{Ejemplos}
%\newtheorem{definicion}{Definici\'on}
\newtheorem{notacion}{Notaci\'on}
\newcommand{\sen}{\operatorname{sen}}
\newcommand{\senh}{\operatorname{senh}}
\newcommand{\csch}{\operatorname{csch}}
\newcommand{\sech}{\operatorname{sech}}
\newcommand{\arcsec}{\operatorname{arcsec}}
\newcommand{\arcsen}{\operatorname{arcsen}}
\newcommand{\id}{\operatorname{id}}
\begin{document}




  \section*{T16 }

  
  
	\begin{enumerate}

        \item Mostrar que $xy+z+3xz^5=4$ es soluble para $z$ como
          funci\'on de $x,y$ cerca de $(1,0,1)$. Adem\'as:
          \begin{enumerate}
          \item calcular $\partial_xz(1,0)$ y $\partial_yz(1,0)$.
          \item usando una aproximaci\'on lineal, estimar el valor de $z$
            cuando $x=1.1$ y $y=0.5$.
          \end{enumerate}
          
        \item Considera la superficie en $\mathbb{R}^3$ dada por
          $$
          2y^2z^2-2x=0
          $$

          \begin{enumerate}

          \item Encuentra dos simetrias de la superficie.
            
          \item Usar el teorema de la funci\'on impl\'icita para verificar
            que, para cualquier punto en la curva,
            podemos despejar $x$ en terminos de $y$ y $z$ y usando
            tambi\'en el teorema de la funci\'on impl\'icita
            calcula $\partial_y x$ y $\partial_z x$
          \item de manera directa despeja $x$ y usa este despeja para calcular
            $\partial_y x$ y $\partial_z x$.

            

          \item Usando el teorema de la funci\'on
            impl\'icita, prueba que cerca de los punto $(1,1,1)$ y $(1,-1,1)$ podemos
            despejar a  $y$, en t\'erminos de las otras dos variables. Adem\'as
            calcula $\partial_xy(1,1)$, $\partial_z y (1,1)$. Nota:
            las parciales dependen si est\'as en el punto $(1,1,1)$ o $(1,-1,1)$.

          \item Trata de despejar de manera directa $y$ en $2y^2z^2-2x=0$
            para encontrar una f\'ormula para $y$ cerca de $(1,1,1)$
            y para $y$ cerca de $(1,-1,1)$.
            

          \end{enumerate}


        \item Analiza la solubilidad del sistema
          \begin{eqnarray*}
            3x+2y+z^2+u+v^2&=&0\\
            4x+3y+z+u^2+v+w+2&=&0\\
            x+z+w+u^2+2&=&0\\
          \end{eqnarray*}
          para $u$, $v$ y $w$ en t\'erminos de $x,y$ y $z$ cerca de $x=y=z=0$,
          $u=v=0$ y $w=-2$.

        \item Investiga si el siguiente sistema

          \begin{eqnarray*}
            u(x,y,z)&=&x+yxz \\
            v(z,y,z)&=&y+xy\\
            w(x,y,z)&=&z+2x+3z^2
          \end{eqnarray*}

          puede resolverse para $x,y,z$ (despejar $x,y$ y $z$) en t\'erminos
          de $u,v$ y $w$, cerca del punto $(x,y,z)=(0,0,0)$.
          
        \item Considera la funci\'on $f:\mathbb{R}\to \mathbb{R}$ dada por
          $$
          f(x)=\left\{
            \begin{array}{cc}
              x+2x^2\sen(1/x) & x\ne 0 \\
              0 & x=0
              \end{array}
            \right.
          $$

          En este caso recuerda que $D_0f$ es simplemente $f'(0)$ y que
          la condici\'on de que $D_0f$ sea inyectiva simplemente se traduce
          a $f'(0)\ne0$.

          Prueba que $f'(0)$ existe, es distinta de cero, pero sin embargo,
          $f$ no es invertible cerca de $x=0$.

          \textquestiondown Contradice esto el teorema de la funci\'on inversa?
            
          \end{enumerate}
          

          

	
  
       \end{document}