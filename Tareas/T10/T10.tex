
\documentclass{article}
\usepackage{amssymb, latexsym, amsmath, amsthm, amsfonts, amsbsy, enumerate, tabularx, graphicx}
\newtheorem{definicion}{Definici\'on}

\newtheorem{teorema}{Teorema}
\newtheorem{prop}{Proposici\'on}
\newtheorem{coro}{Corolario}
\newtheorem{lema}{Lema}
\theoremstyle{definition}
\newtheorem{nota}{Nota}
\newtheorem{notas}{Notas}
\newtheorem{ejemplo}{Ejemplo}
\newtheorem{ejemplos}{Ejemplos}
%\newtheorem{definicion}{Definici\'on}
\newtheorem{notacion}{Notaci\'on}
\newcommand{\sen}{\operatorname{sen}}
\newcommand{\senh}{\operatorname{senh}}
\newcommand{\csch}{\operatorname{csch}}
\newcommand{\sech}{\operatorname{sech}}
\newcommand{\arcsec}{\operatorname{arcsec}}
\newcommand{\arcsen}{\operatorname{arcsen}}
\newcommand{\id}{\operatorname{id}}
\begin{document}

Recuerda que, dado un vector $u\ne 0$, la derivada direccional, a lo largo de $u$, de $f$
en el punto $p$ se define como
$$
\lim_{h \to 0} \frac{f(p+hu)-f(p)}{h}
$$
Se denota $\partial_uf(p)$ o, en las notas en pdf que
se subieron como  $D_uf(p)$. La f\'ormula para calcularlo es
$$
\partial_uf(p)= \langle \nabla_pf, u \rangle 
$$

\section*{T10 }

\begin{enumerate}
\item Para las funciones dadas, calcula: (1o) $\nabla_{(x,y)}f$;
  (2o) $\langle \nabla_{(x,y)} f, u \rangle$;
  (3o) $\partial_{u}f(p)$.

  \begin{enumerate}
  \item $f(x,y)=x^2-y^2$, $u=(\sqrt{3}/2,1/2)$, $p=(1,0)$.
  \item $f(x,y)=e^x\cos(y)$, $u=(0,1)$, $p=(0,\pi/2)$.
  \item $f(x,y)=y^{10}$, $u=(0,-1)$, $p=(1,-1)$.
    \item $f(x,y)=$ distancia de $(x,y)$ a $(0,3)$, $u=(1,0)$, $p=(1,1)$.
  \end{enumerate}

\item Para las siguientes funciones calcula $\nabla_{(x,y,z)}f$ y adem\'as haz un bosquejo
  del campo vectorial $(x,y,z)\mapsto \nabla_{(x,y,z)}f$.
  \begin{enumerate}
  \item $f(x,y,z)=\frac{1}{\sqrt{x^2+y^2+z^2}}$, (fuente saliendo del origen).
  \item $f(x,y,z)=\log(x^2+y^2)$, (fuente axial sobre el eje z).
  \item $f(x,y,z)=\frac{1}{\sqrt{(x-1)^2+y^2+z^2}}-\frac{1}{\sqrt{(x+1)^2+y^2+z^2}}$, (dipolo). 
  \end{enumerate}

\item Para $f(x,y)=3x^2+2y^2$, encuentra la direcci\'on (vector unitario) en la cual la gr\'afica
  de la funci\'on est\'a m\'as empinada  y en la cual se mantiene nivelada, si estas parada
  en el punto $(1,2)$.
  
\item \textquestiondown Cual es el gradiente de $f$,
  una funci\'on de una variable?
  \textquestiondown Cu\'al son las dos posibles direcciones de $u$ al calcular
  la derivada direccional de $f$ a lo largo de $u$ ?

\item Encuentra la direcci\'on $u$ ($\|u\|=1$) para la cual $f$ crece m\'as r\'apidamente
  si est\'as parado en $(1,2)$.
  \begin{enumerate}
  \item $f(x,y)=e^{x-y}$,
  \item $f(x,y)=\sqrt{5-x^2-y^2}$ (cuidado!!!) ,
  \item $f(x,y)=ax+by$.
  \end{enumerate}


\item Asume que $f(x,y)$ y $g(x,y)$ son   funciones, clase $C^1$, tal que
  $\nabla_{(x,y)} f$ es perpendicular a $(3,2)$ con su longitud
  igual a 1 y $\nabla_{(x,y)}g$ es paralelo a $(3,2)$
  con su longitud igual a $5$. Encuentra $\nabla_{(x,y)}f$,
  $\nabla_{(x,y)}g$, $f$ y $g$.

\item Si $f(0,1)=0, f(1,0)=1$ y $f(2,1)=2$, encuentra
  $\nabla_{(x,y)}f$ suponiendo que $f(x,y)=Ax+By+C$.

\item \textquestiondown Qu\'e funciones tienen los siguientes
  gradientes?
  \begin{enumerate}
  \item $(2x+y,x)$,
  \item $(e^{x-y},-e^{x-y})$,
  \item $(y,-x)$.
    
  \end{enumerate}
  


         \item Una funci\'on $f:\mathbb{R}^n \to \mathbb{R}$ se llama par si
           $f(-p)=f(p)$, para toda $p\in \mathbb{R}^n$.

           \begin{enumerate}
         
           \item Sea $f:\mathbb{R} \to \mathbb{R}$ una funci\'on par, de una variable
             y diferenciable en el origen. Demuestra que $f'(0)=0$. 


           \item Sea $f:\mathbb{R}^n \to \mathbb{R}$ una funci\'on par,
             diferenciable en el origen. Demuestra que $\nabla_{0}f=0$.

             Sugerencia: usa el inciso anterior.
             
           \item Sea $g:\mathbb{R}^2\to \mathbb{R}$ una funci\'on diferenciable en
             el origen. Sup\'on que existen dos vectores ortonormales (es decir,
             unitarios y ortogonales) $u$,$v$, tales que
             \begin{eqnarray*}
               g(tu)=g(-tu), \quad g(tv)=g(-tv),\quad \forall t\in \mathbb{R}. 
             \end{eqnarray*}

             Demuestra que $\nabla_{(0,0)}g=0$.
             
             Sugerencia: comienza probando que   $\partial_ug(0,0)=0$ y $\partial_{v}g(0,0)=0$. 

           \end{enumerate}

  
  \end{enumerate}


v  
       \end{document}
