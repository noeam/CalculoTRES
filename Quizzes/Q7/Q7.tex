\documentclass[12pt]{article}
\setlength{\topmargin}{-.75in}\addtolength{\textheight}{2.00in}
\setlength{\oddsidemargin}{.00in}\addtolength{\textwidth}{.75in}
\usepackage{amsmath,amssymb}
\nofiles

\pagestyle{empty}
\setlength{\parindent}{0in}
\newcommand{\sen}{\operatorname{sen}}


\begin{document}

\noindent{\sc {\bf{\large Quiz 7 }}
           \hfill C\'alculo 3, semestre 2020-2}
\bigskip

\noindent {\sc{
            { \Large Nombre: \underline {\hspace {10 cm }}}}}
            
\bigskip
\bigskip
\bigskip


\begin{enumerate}


  

  
\item (3 pts) Usa aproximaciones lineales para estimar las siguientes cantidades. En cada caso
  recuerda escribir la funci\'on que te ayuda  a estimar, el punto alrededor del cual
  se calcula la aproximaci\'on lineal y la aproximaci\'on lineal.

  \begin{enumerate}
  \item(1.5 pts) $\sqrt[3]{(8.1)(26.9)}$.
    \item (1.5 pts) $\sen^2((\pi/2+\pi/10)(1.2))$.
  \end{enumerate}



  \vspace{4cm}

\item (2 pts) Considera la funci\'on $f(x,y,z)=\sen(xy)\cos(yz)$. Dado un punto cualquiera
  $(x_0,y_0,z_0)$ encuentra la f\'ormula de la aproximaci\'on lineal de $f$ (en la notas
  denotada $L(x,y,z)$) alrededor de $(x_0,y_0,z_0)$.
  
  \end{enumerate}


  
\end{document}
