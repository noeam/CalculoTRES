
\documentclass{article}
\usepackage{amssymb, latexsym, amsmath, amsthm, amsfonts, amsbsy, enumerate, tabularx, graphicx}
\newtheorem{definicion}{Definici\'on}

\newtheorem{teorema}{Teorema}
\newtheorem{prop}{Proposici\'on}
\newtheorem{coro}{Corolario}
\newtheorem{lema}{Lema}
\theoremstyle{definition}
\newtheorem{nota}{Nota}
\newtheorem{notas}{Notas}
\newtheorem{ejemplo}{Ejemplo}
\newtheorem{ejemplos}{Ejemplos}
%\newtheorem{definicion}{Definici\'on}
\newtheorem{notacion}{Notaci\'on}
\newcommand{\sen}{\operatorname{sen}}
\newcommand{\senh}{\operatorname{senh}}
\newcommand{\csch}{\operatorname{csch}}
\newcommand{\sech}{\operatorname{sech}}
\newcommand{\arcsec}{\operatorname{arcsec}}
\newcommand{\arcsen}{\operatorname{arcsen}}
\newcommand{\id}{\operatorname{id}}
\begin{document}




  \section*{T9 }


  \begin{enumerate}



  \item\label{Ejer:DesigualdadesAplicandoTVM}
    Utiliza el Teorema de Valor Medio para probar las siguientes desigualdades:
\begin{enumerate}
\item $|\sen(a)-\sen(b)| \leq |a-b|$,
\item $|\sen(2x)-\sen(x)| \leq |x|$.
\end{enumerate}


\item Sean $f,g:[a,\infty)\to \mathbb{R}$, continuas en $[a,\infty)$ y diferenciables en $(a,\infty)$ con
$f(a) \leq g(a)$.

\begin{enumerate}
\item Si $f'(x) < g'(x)$, para todo $x>a$, usa el T.V.M. para demostrar que $f(x) < g(x)$, para toda $x>a$.
\item Usar el inciso anterior para probar que $\sqrt{1+x}< 1+\frac{x}{2}$ para todo $x>0$.
\end{enumerate}    
  
\item Sea $f:B_r(x_0,y_0)\subset \mathbb{R}^2 \to \mathbb{R}$ una funci\'on diferenciable en todo
  punto $(x,y)$, con $(x,y)\in B_r(x_0,y_0)$. Demuestra que, para todos $x_1,x_2$ con $|x_1-x_0|<r$,
  $|x_2-x_0|<r$, se tiene que existe $c$ entre $x_1$ y $x_2$ tal que
  $$
  |f(x_1,y_0)-f(x_2,y_0)| = |\partial_xf(c,y_0)||x_1-x_2|.
  $$

  \item Para las siguientes funciones, usa el criterio
    de las derivadas parciales para encontrar  un dominio
    para el cual la funci\'on sea diferencible en
    todo punto del dominio.
    \begin{enumerate}
    \item $f(x,y)=e^{\sqrt{x^2+y^2}}$,  $(x,y)\in \mathbb{R}^2$,
    \item $f(x,y)=\log(e^{\sqrt[4]{(x-1)^2+(y-2)^2}}+e^{\sqrt{(x-5)^2+(y-6)^2}})$,
      $(x,y)\in \mathbb{R}^2$,
      \item $f(x,y)=\log(1+\sqrt{(x-1)^2+(y+1)^2})$, $(x,y)\in \mathbb{R}^2$.
    \end{enumerate}
    


      \item Usa el teorema de la igualdad de derivadas parciales mixtas
    para dos variables para probar que si $f:\mathbb{R}^3\to \mathbb{R}$
    es de clase $C^3$, entonces
    $$
    \partial^{3}_{xyz}f=\partial_{yzx}^3 f
    $$


    
    \item Para cada una de las siguientes funciones
    calcula $\partial_{x}\partial_yf(x,y)$ y $\partial_y\partial_xf(x,y)$.
    \textquestiondown Qu\'e notas ?
    \begin{enumerate}
    \item $f(x,y)=x^2+xy^2+y^3$
    \item $f(x,y)=\log(x^2+y^4)$
    \item $f(x,y)=e^{3x+2xy+y^2}$                 
    \end{enumerate}

    
  \item Considera  la funci\'on $g(x,t)=2+e^{-t}\sen(x)$,
    $(t,x)\in \mathbb{R}^2$.

    \begin{enumerate}
    \item Demuestra que $g$ satiface la
      ecuaci\'on del calor:
      $$
      \partial_t g= \partial_x^2g.
      $$
      Aqu\'i $g(x,t)$, representa la temperatura de una varilla de metal
      en la posici\'on $x$ al tiempo $t$.
    \item Esbozar la gr\'afica de $g$, para $t\geq 0$.
    \item \textquestiondown Qu\'e sucede con $g(x,t)$ cuando $t\to \infty$ ?
    Interpreta \'este l\'imite en t\'erminos del comportamiento del calor
    en la varilla.

  \end{enumerate}
  

    \begin{definicion}
      Una funci\'on $f:U\to \mathbb{R}$, definida en un abierto,
      se llama arm\'onica en $U$ si las
      derivadas parciales de segundo orden $\partial_x^2 f$ y $\partial_y^2 f$,
      existen en todo punto de $U$, son continuas en todo $U$ y
       $$
       \partial_x^2f(x,y)+\partial_y^2f(x,y)=0
       $$ 
       para todo $(x,y)\in U$. 
     \end{definicion}

  \item  Para las siguientes funciones,
    determina cuales son funciones arm\'onicas. 

       
    \begin{enumerate}
    \item $f(x,y)=x^2-y^2$, $(x,y)\in \mathbb{R}^2$;
    \item $f(x,y)=e^y\cos(x)$, $(x,y)\in \mathbb{R}^2$; 
    \item $f(x,y)=e^y\sen(x)$, $(x,y)\in \mathbb{R}^2$;
    \item $f(x,y)=\log(x^2+y^2)$,
      $(x,y)\in \mathbb{R}^2$, $(x,y)\ne (0.0)$;
    \item $f(x,y)=x^3-3x^2y-3yx^3+y^3$, $(x,y)\in \mathbb{R}^2$.
    \end{enumerate}		       

  \item Demuestra que toda funci\'on
    lineal $f:\mathbb{R}^2\to \mathbb{R}$ es arm\'onica.
    
                
  \item Considera la funci\'on $f(x,y)=ax^2+by^2+cxy$,
    donde $a,b,c\in \mathbb{R}$ son par\'ametros. Encuentra
    los valores de $a$, $b$ y $c$ para los cuales
    \begin{enumerate}
    \item $\partial_x^2 f(x,y)+\partial_y^2f(x,y) >0$, para todo $(x,y)$,
    \item $\partial_x^2f(x,y)+\partial_y^2f(x,y)=0$, para todo $(x,y)$,
    \item $\partial_x^2f(x,y)+\partial_y^2f(x,y)<0$, para todo $(x,y)$.
    \end{enumerate}
                  
 
    
  \item (Vogel) Este ejercicio da un ejemplo
    donde las parciales mixtas no son iguales.

    Considera la funci\'on
    $$
    f(x,y)=\left\{
      \begin{array}{cc}
        \frac{xy^3-x^3y}{x^2+y^2} & (x,y)\ne (0,0)\\
        0 & (x,y)=(0,0)
      \end{array}
    \right.
    $$
    Prueba
    \begin{enumerate}
    \item
      $$
      \partial_xf(x,y)=\left\{
        \begin{array}{cc}
          \frac{y^3-3x^2y}{x^2+y^2}- \frac{2x(xy^3-x^3y)}{(x^2+y^2)^2}
          & (x,y)\ne (0,0)\\
          0 & (x,y)=(0,0)
        \end{array}
      \right.
      $$
    \item
      $$
      \partial_y f(x,y)=\left\{
        \begin{array}{cc}
          \frac{3xy^2-x^3}{x^2+y^2}- \frac{2y(xy^3-x^3y)}{(x^2+y^2)^2}
          & (x,y)\ne (0,0)\\
          0 & (x,y)=(0,0)
        \end{array}
      \right.
      $$
      
    \item $\partial_x \partial_yf(0,0)=-1$ y $\partial_y\partial_xf(0,0)=1$
    \end{enumerate}



    
  \end{enumerate}
  
  
       \end{document}
