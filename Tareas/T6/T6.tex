
\documentclass{article}
\usepackage{amssymb, latexsym, amsmath, amsthm, amsfonts, amsbsy, enumerate, tabularx, graphicx}
\newtheorem{definicion}{Definici\'on}

\newtheorem{teorema}{Teorema}
\newtheorem{prop}{Proposici\'on}
\newtheorem{coro}{Corolario}
\newtheorem{lema}{Lema}
\theoremstyle{definition}
\newtheorem{nota}{Nota}
\newtheorem{notas}{Notas}
\newtheorem{ejemplo}{Ejemplo}
\newtheorem{ejemplos}{Ejemplos}
%\newtheorem{definicion}{Definici\'on}
\newtheorem{notacion}{Notaci\'on}
\newcommand{\sen}{\operatorname{sen}}
\newcommand{\senh}{\operatorname{senh}}
\newcommand{\csch}{\operatorname{csch}}
\newcommand{\sech}{\operatorname{sech}}
\newcommand{\arcsec}{\operatorname{arcsec}}
\newcommand{\arcsen}{\operatorname{arcsen}}
\newcommand{\id}{\operatorname{id}}
\begin{document}




  \section*{T6 }

  \begin{enumerate}

       \item Para cada una de las siguientes funciones encuentra la derivada parcial en los puntos indicados.
       \begin{enumerate}
       \item $f(x,y)=xy+x^2$ en $(1,0)$ y en $(0,1)$.
       \item $f(x,y)=\log(\sqrt{1+x^2y^2})$ en $(1,0)$ y en $(1,1)$.
       \item $f(x,y)=(x^2+y^2)e^{x^2+y^2}$ en $(1,1)$ y en $(2,2)$.
       \item $f(x,y)=xy\cos(2x+y)$ en $(\frac{\pi}{2},0)$ y en $(0,\frac{\pi}{2})$.
       \end{enumerate}
       

     \item Para cada una de las siguientes funciones $f(x,y)$, encuentra el dominio de definici\'on
       de las funciones $\partial_xf(x,y)$ y $\partial_yf(x,y)$.

       \begin{enumerate}
       \item $f(x,y)=\log(x^2+y^2+1)$.
       \item $f(x,y)=\frac{1}{\sqrt{x^2+y^2}}$.
       \item $f(x,y)=y\sqrt{1-x^2-y^2}$.
         \item $f(x,y)=\frac{x}{y}$.
       \end{enumerate}


       
     \item Sea $D:\mathbb{R}\to \mathbb{R}$ la funci\'on de Dirichlet
		$$
		D(x)=\left\{
		\begin{array}{cc}
		1, & \textrm{$x$ racional.}\\
		0, & \textrm{$x$ irracional.}
		\end{array}
		\right.	
		$$
       Define la funci\'on $f:\mathbb{R}^2 \to \mathbb{R}$ dada por $f(x,y)=y^2D(x)$.
       
       \begin{enumerate}
       \item Demuestra que, para todo puntos $(x_0,y_0)$, $\partial_yf(x_0,y_0)$ siempre existe y calcula su valor.
       \item Adem\'as, prueba que si $y_0\ne 0$, $\partial_xf(x_0,y_0)$ no existe y que para todo
       $x_0$, $\partial_yf(x_0,0)=0$. 
       \end{enumerate}

       

     \item Considera la funci\'on $f(x,y)=|x+y|-|x-y|$.
       \begin{enumerate}
       \item Demuestra que, para todos $x_0,y_0\in \mathbb{R}$
         $$
         \partial_xf(x_0,0)=0, \quad \partial_yf(0,y_0)=0.
         $$
       \item Demuestra que $\partial_xf(1,1)$ no existe.
         
       \end{enumerate}


              
     \item Para cada una de las siguientes funciones calcula $\partial_x\left( \int_0^x f(s,y)ds \right)$. \textquestiondown Encuentras alg\'un patron?:
       \begin{enumerate}
       \item $f(x,y)=x^2+xy+y^2$,
       \item $f(x,y)=e^{xy}$,
       \item $f(x,y)=\cos(x+2y)$.
       \end{enumerate}
       
     \item\label{Ejer:TFCyParciales} Para cada una de las siguientes funciones calcula,  $\int_0^x(\partial_s f(s,y))ds$.
      \textquestiondown Encuentras alg\'un patron?

       \begin{enumerate}
       \item $f(x,y)=x^3y^2+xy+x^2y$
       \item $f(x,y)=e^{x^2+y}$
       \item $f(x,y)=\sen(xy)$
       \end{enumerate}

       
      
     \item Este ejercicio generaliza al ejercicio \ref{Ejer:TFCyParciales}. Usa el Teorema Fundamental para probar
       que $\int_0^x(\partial_x f(x,y))dx=f(x,y)-f(0,y)$.
      
       
       

   
     \item Considera la funci\'on
       $$
       f(x,y)=\left\{
         \begin{array}{cc}
           \frac{\sen(x^2+y^2)}{x^2+y^2} & (x,y)\ne (0,0) \\
           1 & (x,y)=(0,0)
         \end{array}
         \right.
         $$
         \begin{enumerate}
         \item Calcula $\partial_xf(0,0)$ y $\partial_yf(0,0)$.
         \item Considera la g\'rfica 2-dimensional que se obtiene al cortar
           la gr\'afica de $f$ con el plano $y=0$. Para \'esta gr\'afica encuentra
           los puntos $(x,z)$,  para los cuales la recta tangente en dicho punto es horizontal.
         \end{enumerate}    
   


         \item Para cada una de las siguientes funciones encuentra la derivada parcial con respecto a la variable indicada.
                \begin{enumerate}
                \item $f(p)=\sum_{i=1}^n p_i^2$. Calcular $\partial_{p_j}f(p)$.
                \item $f(x)=\langle x, y \rangle$, donde $x=(x_1,\dots, x_n)$ y $y=(y_1,\dots, y_n)$
                  es un vector fijo. Calcular
                  $\partial_{x_j}f(x)$. \textquestiondown La notaci\'on $\partial_{y_j}f(x)$ tiene sentido en este ejemplo?
                \item $f(z)=\left( \sum_{i=1}^n \log(z_i^2+1) \right)^2$,
                  donde $z=(z_1,\dots, z_n)$. Calcular $\partial_{z_j}f(z)$.
                \item $f(x,y,z)=100x^{1/2}y^{1/4}z^{1/5}$.
                  Calcular $\partial_xf(x,y,z), \partial_yf(x,y,z)$ y $\partial_zf(x,y,z)$.
                \end{enumerate}
       
                \end{enumerate}
   
       \end{document}
