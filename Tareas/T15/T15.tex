
\documentclass{article}
\usepackage{amssymb, latexsym, amsmath, amsthm, amsfonts, amsbsy, enumerate, tabularx, graphicx}
\newtheorem{definicion}{Definici\'on}

\newtheorem{teorema}{Teorema}
\newtheorem{prop}{Proposici\'on}
\newtheorem{coro}{Corolario}
\newtheorem{lema}{Lema}
\theoremstyle{definition}
\newtheorem{nota}{Nota}
\newtheorem{notas}{Notas}
\newtheorem{ejemplo}{Ejemplo}
\newtheorem{ejemplos}{Ejemplos}
%\newtheorem{definicion}{Definici\'on}
\newtheorem{notacion}{Notaci\'on}
\newcommand{\sen}{\operatorname{sen}}
\newcommand{\senh}{\operatorname{senh}}
\newcommand{\csch}{\operatorname{csch}}
\newcommand{\sech}{\operatorname{sech}}
\newcommand{\arcsec}{\operatorname{arcsec}}
\newcommand{\arcsen}{\operatorname{arcsen}}
\newcommand{\id}{\operatorname{id}}
\begin{document}




  \section*{T15 }

  
  
	\begin{enumerate}


          
        \item Encuentra el m\'inimo de $f(x,y)=xy$ sujeto a la
          restricci\'on $x+y=1$. \textquestiondown Con las
          mismas condiciones, existe el m\'aximo ? Explica.
          
        \item Encuentra los m\'aximos y m\'inimos de la funci\'on
          $f(x,y,z)=x-2y+2z$ sobre la esfera $x^2+y^2+z^2=1$.

          
        \item Encuentra las distancias m\'aximas y m\'inimas 
          desde el origen a la curva $5x^2+6xy+5y^2=8$.

          Sugerencia: es m\'as sencillo minimizar la distancias al cuadrado.


    
        \item Encuentra el m\'inimo de $f(x,y,z)=x^2+2y^2+z^2$ con
          la restricci\'on $x+y+z=0$ y $x-z=1$.

		
		\item Sea $f:\mathbb{R}\to \mathbb{R}$ de la forma $f(t)=mt+c$, donde
		$m$ y $c$ son constantes. Demuestra que el m\'aximo y m\'inimo de
		$f$, restringida al intervalo $[a,b]$ se alcanzan en los extremos del intervalo.
          
        \item Para las siguientes funciones arm\'onicas, encuentra
          el m\'aximo y m\'inimo absolutos sobre la regi\'on dada.

          \begin{enumerate}
         
          \item $f(x,y)=\log(x^2+y^2)$, sobre el anillo
            $S=\{(x,y)\in \mathbb{R}^2:  1\leq  x^2+y^2\leq 4\}$.
          \item $f(x,y)=e^y\cos(x)$, sobre el tri\'angulo (relleno)
            con v\'ertices $(0,\log(5))$, $(-\pi,0)$, $(\pi,0)$.

             \item $f(x,y,z)=x^2-y^2+z$, sobre la esfera
            $S=\{(x,y)\in \mathbb{R}^2: x^2+y^2+z^2=1\}$.
            
          \end{enumerate}
          
        \item Un canal de riego tiene lados y fondo de concreto con
          secci\'on transversal trapezoidal de \'area $A=y(x+y\tan(\theta))$
          y per\'imetro h\'umedo $P=x+\frac{2y}{\cos(\theta)}$, donde
          $x$ es el ancho del fondo, $y$ la profundidad del agua y
          $\theta$ la inclinaci\'on lateral, medida a partir de la
          vertical. El mejor dise\~no para una inclinaci\'on
          fija $\theta$ se halla resolviendo $P=$ m\'inimo
          sujeto a la condici\'on $A=$ constante. Mostrar que
          $y^2=\frac{A\cos(\theta)}{2-\cos(\theta)}$.

        \item Considera la funci\'on $f(x,y)=x^2+xy+y^2$,
          en el disco unitario $D=\{(x,y): x^2+y^2 \leq 1\}$. Usa
          multiplicadores de Lagrange para maximizar $f$ en
          el circulo unitario  $\partial D=\{(x,y): x^2+y^2=1\}$. Usa
          lo anterior para encontrar los m\'aximos y m\'inimos absolutos
          de $f$ en $D$.
          
        \item Minimiza la funci\'on $f(x,y)=x^2+xy+y^2$ con
          la restricci\'on $x+y \geq 4$.

          
          
        \item Maximiza la funci\'on $f(x,y,z)=2x+3y+5z$ sujeto
          a las restricciones $x\geq 0, y\geq 0, z\geq 0 $ y
          $x+y+z \leq 1$.


        \item 
          \begin{enumerate}
          \item Encuentra m\'aximo de la funci\'on
            $f(x_1,\dots, x_n)=(x_1\cdots x_n)^2$, sujeta a la
            restricci\'on $x_1^2+\cdots +x_n^2=1$.
          \item Usando el inciso anterior demuestra la
            desigualdad aritm\'etico geom\'etrica, es decir,
            para escalares mayores o iguales a cero, $a_1,\dots, a_n$,
            se cumple
            $$
            \sqrt[n]{a_1\cdots a_n} \leq \frac{a_1+\cdots +a_n}{n}
            $$

            Sugerencia: considera
            $x_i=\frac{\sqrt{a_i}}{\sqrt{a_1+\cdots +a_n}}$, $i=1,\dots,n$.
          \end{enumerate}

        \item
          \begin{enumerate}
          \item Sean $p>1, q>1$ tal que $\frac{1}{p}+\frac{1}{q}=1$. Demuestra
            que el m\'inimo de la funci\'on
            $$
            f(x,y)=\frac{x^p}{p}+\frac{y^q}{q}
            $$
            sujeta a la restricci\'on $xy=1$ es 1.

          \item Usando el inciso anterior, prueba que si $a$ y $b$ son
            reales mayores o iguales a cero, entonces
            $$
            ab \leq \frac{a^p}{p}+\frac{b^q}{q}
            $$
          \item Sean $a_1,\dots, a_n$, $b_1,\dots, b_n$, numeros
            reales mayores o iguales a cero, demuestra la desigualdad
            $$
            \sum_{j=1}^n a_jb_j \leq \left( \sum_{j=1}^n a_j^p \right)^{1/p}
            \left( \sum_{j=1}^n b_j^q \right)^{1/q}
            $$

            Sugerencia: sean $A=\left( \sum_{j=1}^n a_j^p \right)^{1/p}$,
            $B=\left( \sum_{j=1}^n b_j^q \right)^{1/q}$, aplica
            la desigualdad del inciso anterior a $a=a_j/A$, $b=b_j/B$.
         
          \end{enumerate}
          
            
          \end{enumerate}
          

          

	
  
       \end{document}